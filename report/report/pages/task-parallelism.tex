Dalším způsobem, jak efektivněji implementovat stejný algoritmus, bylo použití taskového paralelismu, který pro urychlení výpočtu používá více vláken.
Základní princip taskového paralelismu spočívá v tom, že nejprve vlákno vytvoří úlohu, kterou umístí do task poolu, a poté ji další vlákno z task poolu vezme a provede.
Úlohy jsou vkládány do task poolu více vlákny a jsou také z něho odebírány více vlákny.

Sekvenční algoritmus lze paralelizovat, jelikož jsou všechny stavy ve vyhledávacím prostoru na sobě nezávislé.
K implementaci taskového paralelismu byla použita knihovna \textbf{OpenMP}, která k paralelizaci algoritmu používá \textit{pragma} direktivy.
Třída \texttt{finder::task\_parallel} implementuje paralelní algoritmus.
Má podobné rozhraní jako třída \texttt{finder::sequential}.

Kromě grafu se konstruktoru předává i poměr sekvenčně prohledávaných stavů.
Z poměru se vypočítá hloubka, od které jsou stavy prohledávány sekvenčně.
Pokud je hodnota poměru jedna, pak se prohledávají sekvenčně všechny stavy, na druhou stranu, pokud je hodnota nula, pak žádný z nich.

První velký rozdíl v implementaci byl v metodě \texttt{find} (viz \ref{lst:find}).
Počáteční volání metody rekurzivního vyhledávání bylo obaleno omp direktivami.
Direktiva \textbf{omp parallel} označuje oblast, která je vykonávána více vlákny.
Direktiva \textbf{omp master} určuje oblast, kterou provádí pouze hlavní vlákno, což je velmi důležité, jinak se celé vyhledávání provede tolikrát, kolik je vláken.

\begin{lstlisting}[language=C++, label={lst:find}, title={Metoda pro nalezení nejlepšího stavu}]
    state find(state init)
    {
        ...

        // find best state
        #pragma omp parallel
        #pragma omp master
        bb_dfs(init);

        return best_;
    }
\end{lstlisting}

V samotné metodě \texttt{bb\_dfs} se pár věcí změnilo.
Nejprve volá metodu \texttt{try\_update\_best} (viz \ref{lst:try_update_best}, která se snaží aktualizovat nejlepší stav.
Implementace této metody byla obalena direktivou \textbf{omp critical}, která poskytuje výhradní přístup do dané oblasti právě jednomu vláknu.

\begin{lstlisting}[language=C++, label={lst:try_update_best}, title={Metoda pro pokus o aktualizaci nejlepšího stavu}]
    void try_update_best(const state& candidate)
    {
        #pragma omp critical
        {
            if (candidate.n_colored()==graph_.n_vertices() && candidate.subgraph_connected()
                    && best_.total_weight()<candidate.total_weight()) {
                best_ = candidate;
            }
        }
    }
\end{lstlisting}

Uvnitř hlavní smyčky je volána metoda \texttt{select\_edge} (viz \ref{lst:select_edge}, která musela být také změněna.
Pokud se rozhodne, že hranu lze vybrat a hloubka nedosáhla sekvenční hloubky, tak se úloha přidá do task poolu pomocí direktivy \textbf{omp task}.

\begin{lstlisting}[language=C++, label={lst:select_edge}, title={Metoda pro označení hrany}]
    void select_edge(color from, color to, state curr)
    {
        ...

        // can be added
        if ((curr_from==from || curr_from==colorless)
            && (curr_to==to || curr_to==colorless)) {
            ...
            
            if (graph_.n_edges()-curr.edge_idx()-1>n_sequential_) {
                #pragma omp task
                bb_dfs(curr);
            }
            // search sequentially
            else {
                bb_dfs(curr);
            }
        }
    }
\end{lstlisting}

V samotné metodě (viz \ref{lst:task:bb_dfs}) byl přidán pouze jeden řádek s direktivou \textbf{omp taskwait}, který čeká na dokončení vytvořených úloh, aby mohla pokračovat v prohledávání.

\begin{lstlisting}[language=C++, label={lst:task:bb_dfs}, title={Metoda pro prohledávání stavového prostoru}]
    void bb_dfs(state curr)
    {
        try_update_best(curr);

        while (curr.edge_idx_<graph_.n_edges()) {
            ...

            select_edge(green, red, curr);

            select_edge(red, green, curr);

            #pragma omp taskwait

            ...
        }
    }
\end{lstlisting}

K výsledkům zapsaným do konzole (viz \ref{lst:task:console_log}) byl přidán počet vláken a poměr sekvenčně prohledávaných stavů.
Počet vláken byl nastaven voláním funkce \texttt{omp\_set\_num\_threads} před voláním metody \texttt{find}.

\begin{lstlisting}[language=bash, label={lst:task:console_log}, title={Příklad výpisu výsledků}]
filename: "graf_10_3.txt"
n vertices: 10
n edges: 15
sequential ratio: 0.5
n threads: 2
res:
duration: 0.000385726
best:
vertex_colors: 0,1,0,0,1,1,0,0,1,1
selected_edges: 1,1,1,1,0,1,1,1,1,0,1,1,1,1,1
total_weight: 1300
edge_idx: 15
--------------------------------------------------
\end{lstlisting}