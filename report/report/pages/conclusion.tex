Vezmeme-li tyto maličkosti stranou, zdá se, že implementace byly provedeny kvalitně.
Algoritmus lze vždy urychlit psaním kódu v \texttt{C} namísto \texttt{C++}, ale jsem vděčný, že jsme nemuseli.
Bylo dobře, že díky jednoduchosti algoritmu bylo možné věnovat více úsilí samotnému zrychlování a seznámení se s OpenMP a MPI.
Další ulehčení bylo, když jsme na začátku semestru obdrželi video s nápovědou, jak implementovat sekvenční algoritmus.
Dokumentace zadání byla dobře napsaná a pomohla s implementací algoritmu.
Testovací instance byly připraveny správně, byly snadno dostupné a užitečné ve fázi ladění.
Dokumentace pro generátor grafů byla dobře zpracovaná a užitečná ve fázi hodnocení projektu.
Detaily skriptů a informace o tom, jak je zkompilovat a spouštět naše projekty, byly také provedeny dobře, ale možná si zaslouží svou vlastní sekci na webu s názvem STAR.
Další skvělá věc byla, že si k nám cvičící při každém cvičení sedl a konzultoval s námi naše projekty.
Celkově měl člověk možnost naučit se něco nového a soustředit se na to nejdůležitější díky kvalitně zpracované dokumentaci.
