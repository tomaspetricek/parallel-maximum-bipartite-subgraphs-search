Nejprve bylo potřeba načíst graf pomocí funkce \texttt{read\_graph}.
Pro uložení grafu byla zvolena datová struktura \textbf{seznam hran} reprezentovaná třídou \texttt{graph::edge\_list}.
Vzhledem k tomu, že počet hran není v době kompilace znám, byl použit \texttt{std::vector} pro ukládání hran.
Hrany byly reprezentovány pomocí struktury \texttt{graph::edge}, která si ukládá váhu a indexy vrcholů, které spojuje.

Poté byl graf předán do konstruktoru třídy \texttt{finder::sequential}, která implementuje sekvenční algoritmus.
Účelem třídy je najít nejlepší stav.
Každý podgraf je reprezentován stavem ve vyhledávacím prostoru.
Jedinečně se vyznačuje seznamem vybraných hran.
Každý stav je reprezentován třídou \texttt{finder::state}, která si kromě seznamu vybraných hran uchovává i obarvení jednotlivých vrcholů, podgraf, potenciální váhu, index hrany, se kterou aktuálně pracuje a další informace.
Podgraf je reprezentován třídou \texttt{graph::adjacency\_list} a používá se pouze ke kontrole, zda jsou všechny hrany v grafu spojené.
Pro kontrolu se používá algoritmus \textbf{prohledávání do hloubky} (DFS), a proto je graf uložen pomocí datové struktury seznamu sousedů.
Má časovou složitost \(\mathcal{O}(V+E)\), kde \(V\) je počet vrcholů a \(E\) je počet hran.

Pro samotné hledání nejlepšího stavu, byl použit algoritmus \textbf{prohledávání do hloubky s metodou větví a hranic} (BB-DFS).
Vyhledávání pouze pomocí DFS má časovou složitost \(\mathcal{O}(3^n)\), což řádí problém do kategorie \textit{NP-těžkých} problémů.
Účelem metody BB je snížit složitost tím, že se nebudou prohledávat stavy, které nemohou být lepší než již nalezený nejlepší stav (ořezávání shora).
Metoda sama o sobě řádově zkracuje čas na vyřešení problému.

Sekvenční algoritmus začíná řazením hran grafu podle jejich váhy v sestupném pořadí.
To také zrychluje vyhledávání, protože metoda BB může lépe (dříve) prořezávat stavový prostor.
Jeden vrchol je pak obarven, což snižuje počet vyhledávacích stavů, jelikož tím zabraňuje vybarvování v obráceném pořadí.

Dále začne prohledávání.
Metoda \texttt{bb\_dfs} pro vyhledávání se volá rekurzivně.
Při každém volání se snaží aktualizovat nejlepší stav a následně vstoupí do smyčky, která se pokouší přidat hranu do podgrafu.
Existují dva způsoby, jak přidat hranu do podgrafu, protože existují dva způsoby obarvení.
Pokud lze hranu přidat, je zahájeno další volání rekurzivní metody.
Jestli byly vyzkoušeny oba způsoby barvení, vyzkouší se poslední stav a to znamená, že se hrana vůbec nepřidá a je započata další iterace.
Smyčka končí, jakmile index hrany dosáhne počtu hran grafu.
Důležitým detailem je, že aktuální stav je zkopírován pokaždé, když je předán rekurzivní metodě, což usnadňuje paralelizaci v dalších implementacích.

\begin{lstlisting}[language=C++, label={lst:bb_dfs}, title={Metoda pro prohledávání stavového prostoru}]
    void bb_dfs(state curr)
    {
        try_update_best(curr);

        while (curr.edge_idx_<graph_.n_edges()) {
            // check upper bound
            if (!can_be_better(best_, curr))
                return;

            // update potential weight
            curr.potential_weight_ += graph_.edge(curr.edge_idx()).weight;

            select_edge(green, red, curr);

            select_edge(red, green, curr);

            // update index
            curr.edge_idx_++;
        }
    }
\end{lstlisting}

Zda má současný stav potenciál být lepší než nejlepší stav, se vypočítá na základě jeho aktuální váhy, potenciální váhy a celkové váhy grafu.
Potenciální váha podgrafu je váha, kterou by graf měl, kdyby do něj byly přidány všechny hrany, které mohly být v daném stavu přidány.
Vzorec použitý pro porovnání stavů je vidět na implementaci metody \texttt{can\_be\_better}.

\begin{lstlisting}[language=C++, label={lst:can_be_better}, title={Metoda pro ořezávání shora}]
    bool can_be_better(const state& best, const state& other)
    {
        int max_weight = other.total_weight()+
                (graph_.total_weight()-other.potential_weight());
        return max_weight>best.total_weight();
    }
\end{lstlisting}

Jakmile je nalezen nejlepší stav, výsledky programu jsou vypsány do konzole.
Příklad výpisu (viz \ref{lst:seq:console_log}) ukazuje, že jsou zobrazeny informace, jako je název souboru, rozměry grafu, výpočetní čas a atributy nejlepšího stavu.
Celková váha podgrafu poskytnutá programem byla porovnána s referenční celkovou váhou ze zadání.
Srovnání bylo provedeno pro každý testovací graf, aby byla ověřena správnost programu.

\begin{lstlisting}[language=bash, label={lst:seq:console_log}, title={Příklad výpisu výsledků}]
filename: "graf_10_3.txt"
n vertices: 10
n edges: 15
res:
duration: 0.00066782
best:
vertex_colors: 0,1,0,0,1,1,0,0,1,1
selected_edges: 1,1,1,1,0,1,1,1,1,0,1,1,1,1,1
total_weight: 1300
edge_idx: 15
--------------------------------------------------
\end{lstlisting}






