Nejprve bylo potřeba načíst graf pomocí funkce \texttt{read\_graph}.
Pro uložení grafu byla zvolena datová struktura \textbf{seznam hran} reprezentovaná třídou \texttt{graph::edge\_list}.
Vzhledem k tomu, že počet hran není v době kompilace znám, byl použit \texttt{std::vector} pro ukládání hran.
Hrany byly reprezentovány pomocí struktury \texttt{graph::edge}, která si ukládá váhu a indexy vrcholů, které spojuje.

Poté byl graf předán do konstruktoru třídy \texttt{finder::sequential}, která implementuje sekvenční algoritmus.
Účelem třídy je najít nejlepší stav.
Každý podgraf je reprezentován stavem ve vyhledávacím prostoru.
Jedinečně se vyznačuje seznamem vybraných hran.

Každý stav je reprezentován třídou \texttt{finder::state}, která si kromě seznamu vybraných hran uchovává i obarvení jednotlivých vrcholů, podgraf, potenciální váhu, index hrany, se kterou aktuálně pracuje a další informace.

Podgraf je reprezentován třídou \texttt{graph::adjacency\_list} a používá se pouze ke kontrole, zda jsou všechny hrany v grafu spojené.
Pro kontrolu se používá algoritmus \textbf{hledání do hloubky} (DFS), a proto je graf uložen pomocí datové struktury seznamu sousedů.
Má časovou složitost \(\mathcal{O}(V+E)\), kde \(V\) je počet vrcholů a \(E\) je počet hran.

Pro samotné hledání nejlepšího stavu, byl použit algoritmus \textbf{prohledávání do hloubky s metodou větví a hranic} (BB-DFS).
Vyhledávání pouze pomocí DFS má časovou složitost \(\mathcal{O}(3^n)\), což řádí problém do kategorie NP-těžkých problémů.
Účelem metody BB je snížit složitost tím, že se nebudou prohledávat stavy, které nemohou být lepší než již nalezený nejlepší stav (ořezávání shora).
Metoda sama o sobě řádově zkracuje čas na vyřešení problému.
Metoda \texttt{can\_be\_better} byla použita k určení, zda může být daný stav lepší než nejlepší stav.

\begin{lstlisting}[language=C++, label={lst:can_be_better}, title={Metoda pro ořezávání shora}]
    bool can_be_better(const state& best, const state& other)
    {
        int max_weight = other.total_weight()+
                (graph_.total_weight()-other.potential_weight());
        return max_weight>best.total_weight();
    }
\end{lstlisting}




