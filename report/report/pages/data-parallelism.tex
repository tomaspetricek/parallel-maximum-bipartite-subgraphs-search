Dalším způsobem paralelizace programu bylo využití datového paralelismu, který používá také více vláken, ale jiným způsobem.
Pro funkci, jejíž volání jsou na sobě nezávislá, je připraveno několik instancí vstupních dat.
Volání funkce jsou rozdělena mezi vlákna, která jsou umístěna v \textit{thread poolu}.
Pro implementaci datového paralelismu byla opět použita knihovna \textbf{OpenMP}.
Třída \texttt{finder::data\_parallel} implementuje paralelní algoritmus.
Kromě grafu bere konstruktor jako parametr instanci třídy \texttt{finder::explorer}, která se používá pro sběr počátečních stavů. 

Pro vytvoření instance vyžaduje třída \texttt{finder::explorer}  právě jeden parametr, který určuje hloubku, ve které se nacházejí počáteční stavy.
Hloubka se může pohybovat od nuly, což znamená, že by byl shromážděn pouze kořenový stav, až po počet hran grafu, což by znamenalo, že by byly shromážděny všechny stavy.

Maximální počet počátečních stavů lze určit umocněním čísla 3 na hodnotu hloubky, což je dáno způsobem prohledávání vyhledávacím prostorem.
Počet počátečních stavů je ve skutečnosti mnohem nižší, protože při sběru byla také použita metoda větví hranic.
Každý počáteční stav vytváří podprostor, jehož tvar tvoří ternární strom a maximální počet stavů, které lze v každém z nich prozkoumat, je stejný.
Důležitou vlastností podprostorů je, že jsou navzájem \textit{disjunktní}, což znamená, že každý stav může být prohledán právě jednou.

Interně si ukládá počáteční stavy do \texttt{std::vector}, ke kterému lze přistupovat pomocí getteru \texttt{states}.
Hlavní metoda volaná při shromažďování počátečních stavů je \texttt{keep\_exploring} (viz \ref{lst:data:keep_exploring}), která přidá stav k počátečním stavům, pokud bylo dosaženo dané hloubky, a vrátí false pro zastavení hledání v dané větvi, nebo vrací true pro pokračování hledání.

\begin{lstlisting}[language=C++, label={lst:data:keep_exploring}, title={Metoda pro řízení prohledávání}]
    bool keep_exploring(const finder::state& state)
    {
        if (state.edge_idx()==max_depth_) {
            states_.emplace_back(state);
            return false;
        }
        else {
            return true;
        }
    }
\end{lstlisting}

Třída \texttt{finder::data\_parallel} používá metodu \texttt{prepare\_states} k přípravě počátečních stavů pro paralelní oblast.
Zde se používá instance třídy \texttt{finder::explorer}.
Nejprve je převeden na ukazatel, aby mohl simulovat volitelný parametr metody \texttt{bb\_dfs}.

\begin{lstlisting}[language=C++, label={lst:data:prepare_states}, title={Metoda pro přípravu počátečních stavů}]
    std::vector<state> prepare_states(state init)
    {
        ...

        auto expl = std::make_unique<explorer>(explorer_);
        // prepare states
        bb_dfs(init, expl.get());
        return expl->states();
    }
\end{lstlisting}

Jediné, co se v metodě \texttt{bb\_dfs} (viz \ref{lst:data:bb_dfs}) změnilo, bylo to, že před prozkoumáním nového stavu zkontroluje, zda má pokračovat v hledání.
Metoda \texttt{try\_update\_best} je umístěna do paralelní oblasti stejným způsobem jako při implementaci taskového paralelismu.

\begin{lstlisting}[language=C++, label={lst:data:bb_dfs}, title={Metoda pro prohledávání stavového prostoru}]
        void bb_dfs(state curr, explorer* explorer = nullptr)
        {
            try_update_best(curr);
            
            while (curr.edge_idx_<graph_.n_edges()) {
                if (explorer)
                    if (!explorer->keep_exploring(curr))
                        return;
                ...
            }
        }
\end{lstlisting}

Metoda \texttt{find} nejprve shromáždí počáteční stavy pomocí metody \texttt{prepare\_states} a poté spustí datový paralelismus.
Používá direktivu \texttt{omp parallel for} ke spuštění paralelního cyklu for.
Jedna věc, kterou je třeba poznamenat, je, že metoda \texttt{bb\_dfs} se volá bez parametru explorer.

\begin{lstlisting}[language=C++, label={lst:data:find}, title={Metoda pro nalezení nejlepšího stavu}]
    state find(const state& init)
    {
        std::vector<state> states = prepare_states(init);

        // find best state
        #pragma omp parallel for
        for (int i = 0; i<states.size(); i++) {
            bb_dfs(states[i]);
        }

        return best_;
    }
\end{lstlisting}

Maximální hloubka, ve které se nacházejí počáteční stavy, byla přidána do výstupu konzole.

\begin{lstlisting}[language=bash, label={lst:data:console_log}, title={Příklad výpisu výsledků}]
filename: "graf_10_3.txt"
n vertices: 10
n edges: 15
max depth: 5
n threads: 5
res:
duration: 0.000411941
best:
vertex_colors: 0,1,0,0,1,1,0,0,1,1
selected_edges: 1,1,1,1,0,1,1,1,1,0,1,1,1,1,1
total_weight: 1300
edge_idx: 15
--------------------------------------------------
\end{lstlisting}









